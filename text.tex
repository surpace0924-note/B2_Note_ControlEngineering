\documentclass[autodetect-engine,dvipdfmx-if-dvi,ja=standard]{bxjsarticle}

% 二段組にするとき
% \documentclass[twocolumn,autodetect-engine,dvipdfmx-if-dvi,ja=standard]{bxjsarticle}

\usepackage{graphicx}        %図を表示するのに必要
\usepackage{color}           %jpgなどを表示するのに必要
\usepackage{amsmath,amssymb} %数学記号を出すのに必要
\usepackage{setspace}
% \usepackage{eclclass}
\usepackage{cases}
\usepackage{here}
\usepackage{fancyhdr}
\usepackage{ascmac}
\usepackage{lscape}
\usepackage{titlesec}

% 文書全体のスタイルを設定(主に余白)
\setlength{\topmargin}{-0.3in}
\setlength{\oddsidemargin}{0pt}
\setlength{\evensidemargin}{0pt}
\setlength{\textheight}{44\baselineskip}

\titleformat*{\subsection}{\normalsize\bfseries}

% 行頭の字下げをしない
\parindent = 0pt

% ヘッダとフッタの設定
\lhead{制御工学}
\chead{期末試験対策}
\rhead{}
\lfoot{}
\cfoot{-\thepage-} % ページ数
\rfoot{}

% 式の番号を(senction_num.num)のようにする
\makeatletter
\@addtoreset{equation}{section}
\def\theequation{\thesection.\arabic{equation}}
\makeatother

% 画像の貼り付けを簡単にする
\newcommand{\pic}[2]
{
  \begin{figure}[H]
    \begin{center}
      \includegraphics[scale=#2]{#1}
    \end{center}
  \end{figure}
}

% 単位の記述を簡単にする
\newcommand{\unit}[1]
{
  \, [\mathrm{#1}]
}
\begin{document}
% \maketitle
\pagestyle{fancy}
\section{説明問題}
\subsection{航空機の制御システムを設計するうえで重要なことを挙げよ}
 「飛行機を墜落させない」という全体を俯瞰する視点を欠かさない.\\
 プログラムが判断をするとしても,最終決定を下すのは人間である必要がある.

\subsection{Hapticsとは何か}
 利用者に力、振動、動きなどを与えることで皮膚感覚フィードバックを得る技術.\\

\subsection{アクティブ制振制御の原理と応用例について述べよ}
 最上部の重りで塔の揺れと反対方向の力を発生させ振動を抑制する.\\
 東京スカイツリーや高層ビルに使われる.

\subsection{アクティブサスペンションの原理と応用例について述べよ}
 サスペンションにフィードバック制御をかけることで乗り心地や操作性を高めている.\\
 新幹線では車体に備えられた加速度計で横揺れを検知し,横揺れを打ち消すようにコンピュータで振動とは逆方向の力をアクチュエータで生成している.

\subsection{制振と制御対象}
 破壊の原因となるため,制御対象の限界を超えた制御をしてはいけない.\\
 不安定動作の原因となるため,制御対象の特性がわからないのに、無理やり制御ループの中に入れて制御してはいけない.

\subsection{地震に対する耐震,免振と制振の違いについて述べよ}
 耐震:建物を振動に強い作りにすること.\\
 免振:地面から絶縁することで建物に揺れが伝わらないようにする.\\
 制振:振動の衝撃を吸収する装置を取り付け,振動の影響を抑制する.

\subsection{アクティブノイズキャンセリングの原理について述べよ}
 マイクをでノイズを取り込み解析し,逆極性の波をリアルタイムで流すことで,低周波のノイズを消す.ヘッドホンや排気ダクト等の騒音低減に用いられている.

\subsection{大森公式}
 地震で初期微動継続時間から震源距離(観測地点から震源までの距離)を求める式.\\
 地震波の伝搬速度を光波と音波の伝搬速度に置き換えることで,落雷地点からの距離を求めることにも応用できる.

\newpage
\subsection{六本木ヒルズ回転ドア事故から学ぶべき教訓を述べよ}
 暗黙知であった「ドアは軽くてゆっくり動かなければ危ない」という「本質安全」をないがしろにして,「制御安全」だけを考えて設計されてしまった.\\
 「本質設計」として最低限,安全な設計をする.つまり,バグが出ても最悪のことが起こらないような設計をする.

\subsection{10ジュール則とは何か.簡単に述べよ}
 ドアの移動質量が持っているエネルギーが10ジュールを超えると,人間に負傷させる可能性がある.そこで安全対策のために,ドアの移動質量が持つエネルギーは10J以下に抑えて設計するべきという暗黙の了解.

\subsection{ハインリッヒの法則}
 1:29:300

\vspace{7cm}

\end{document}
